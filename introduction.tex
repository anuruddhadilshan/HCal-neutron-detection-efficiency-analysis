\section{Introduction}

The so called ``End-Point" technique has been proposed to be used for the neutron detection efficiency calibration of the BigHAND nucleon detector in the original GMn proposal \cite{gmn_original_proposal}. With the construction of HCal for SBS experiments, which is by design a hadron calorimeter that should give identical (and very high) detection efficiencies for both protons and neutrons, these dedicated proton and neutron calibration runs were dropped during the actual SBS GMn experiment run which took place during October 2021 through February 2022.

However, a precise knowledge of the HCal neutron and proton detection efficiencies is crucial, especially for the analysis of GMn/nTPE experiments. Therefore, an effort is underway to revisit the ``End-Point" technique, described in the original GMn proposal to obtain a clean tagged neutron source from the reaction $p(\gamma,\pi^+)n$ for neutron detection efficiency calibration of HCal. We are currently looking at the LH2 data from the SBS-9 kinematics since it has the lowest momenta for $\pi^+$, which makes applying the end-point cuts to separate events from the three-body and higher pion photo-production channels, $p(\gamma,\pi^+)N\pi$, feasible with the 1.5 \% worst case momentum resolution of BigBite. We are also planning to analyze SBS-4 LH2 data, and also planning on a dedicated run for calibration with more optimal conditions (BigBite in reverse polarity and having a copper radiator in front of the LH2 target for efficient photon production), during the upcoming GEn-RP run in the spring of 2024. 